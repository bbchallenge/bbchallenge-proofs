% !TeX root = ../correctness-deciders.tex

\section{Bouncers}\label{sec:bouncers}

Intuitively, a \emph{bouncer} is a Turing machine that builds a tape out of
ever-expanding repeating fragments. The decider analyses the machine to
identify the growing \emph{repeaters} and the \emph{walls} that separate them,
and then executes the machine symbolically to prove that this growth is endless.

\paragraph{Notation.} For a tape segment $s \in \{0, 1\}^*$, we will
write $(s)$ to mean $s$ repeated any number of times, i.e.~$s^k$ for
an unspecified $k$.

\begin{example}
Consider machine \#88,427,177 (see \url{https://bbchallenge.org/88427177}).
After executing it for 25 steps, we reach the configuration
\begin{equation}
    0^\infty\; 111\; 0\; 00 \rhead D 0^\infty.
\end{equation}
Note that this matches the symbolic tape
\begin{equation}
    \label{eqn:initial-symbolic}
    0^\infty\; (111)\; 0\; (11)\; 00 \rhead D 0^\infty.
\end{equation}
Straightforward simulation for 25 more steps yields
\begin{equation}
    0^\infty\; (111)\; 0\; (11) \lhead A 0101011\; 0^\infty.
\end{equation}
We note that the subtape $11 \lhead A$ turns into $\lhead A 01$ in 2 steps. By repeating
this an appropriate number of times, we can reach the configuration
\begin{equation}
    0^\infty\; (111)\; 0 \lhead A (01)\; 0101011\; 0^\infty.
\end{equation}
One step later, we get
\begin{equation}
    0^\infty\; (111)\; 1 \rhead B (01)\; 0101011\; 0^\infty.
\end{equation}
We would now presume that the subtape $\rhead B 01$, or at the very least $1 \rhead B 01$,
would evolve into something
useful. However, since it doesn't enter state $B$ again, we have no hope of
handling an arbitrary number of repeating $(01)$ in a generic manner.

To make progress, we must instead rewrite the tape as
\begin{equation}
    0^\infty\; (111)\; 1 \rhead B 010101\; (01)\; 1\; 0^\infty.
\end{equation}
After 12 steps of simulation, we reach
\begin{equation}
    0^\infty\; (111)\; 1110110 \rhead D (01)\; 1\; 0^\infty.
\end{equation}
Now, being in state $D$ and with a 0 to the left of the tape head, we have
$0 \rhead D 01 \rightarrow 110 \rhead D$, and so we reach
\begin{equation}
    0^\infty\; (111)\; 111011\; (11)\; 0 \rhead D 1\; 0^\infty,
\end{equation}
and after one step, we have
\begin{equation}
    0^\infty\; (111)\; 111011\; (11)\; 00 \rhead D 0^\infty.
\end{equation}
Since this is a special case of \eqref{eqn:initial-symbolic}, we can
conclude that this machine will never halt. \qed

\end{example}

\todo{The below is not yet coherent, but merely snippets of what might
be useful, created through endless rewriting}
\begin{remark} The symbolic tapes we're considering can also be understood as
defining a regular language of tapes matching a particular step of the cycle
being simulated. In this interpretation, we are executing the Turing machine
on all elements of the language at the same time.
\end{remark}

We will be considering a tape language of the form
\begin{equation}
    \{
    0^\infty\; [0]_q\; w_0\; (r_1)^{k_1}\; w_1\; (r_2)^{k_2}\; w_2\: \cdots\: (r_n)^{k_n}\; w_n
    \mid k_1, k_2, \ldots, k_n \in \N
    \}.
\end{equation}
for a fixed list of \emph{walls} $w_0,\ \ldots,\ w_n \in \{0, 1\}^*$,
a list of \emph{repeaters} $r_1,\ \ldots,\ r_n$, and a fixed state $q$.
