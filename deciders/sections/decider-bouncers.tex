% !TeX root = ../correctness-deciders.tex

\section{Bouncers}\label{sec:bouncers}

Intuitively, a \emph{bouncer} is a Turing machine that builds a tape out of
ever-expanding repeating fragments. The decider analyses the machine to
identify the growing \emph{repeaters} and the \emph{walls} that separate them,
and then executes the machine symbolically to prove that this growth is endless.

\paragraph{Notation.} For a tape segment $s \in \{0, 1\}^*$, we will
write $(s)$ to mean $s$ repeated any number of times, i.e.~$s^k$ for
an unspecified $k$.

\begin{example}
Consider machine \#88,427,177 (see \url{https://bbchallenge.org/88427177}),
which the bouncer decider can prove to never halt. We shall now retrace the
steps it takes, to understand the proof approach.
After executing 25 steps from the empty tape, we reach the configuration
\begin{equation}
    0^\infty\; 111\; 0\; 00 \rhead D 0^\infty.
\end{equation}
Note that this matches the symbolic tape
\begin{equation}
    \label{eqn:initial-symbolic}
    0^\infty\; (111)\; 0\; (11)\; 00 \rhead D 0^\infty.
\end{equation}
Straightforward simulation for 25 more steps yields
\begin{equation}
    0^\infty\; (111)\; 0\; (11) \lhead A 0101011\; 0^\infty.
\end{equation}
We note that the subtape $11 \lhead A$ turns into $\lhead A 01$ in 2 steps. By repeating
this an appropriate number of times, we can reach the configuration
\begin{equation}
    0^\infty\; (111)\; 0 \lhead A (01)\; 0101011\; 0^\infty.
\end{equation}
One step later, we get
\begin{equation}
    0^\infty\; (111)\; 1 \rhead B (01)\; 0101011\; 0^\infty.
\end{equation}
We would now presume that the subtape $\rhead B 01$, or at the very least $1 \rhead B 01$,
would evolve into something
useful. However, since it doesn't enter state $B$ again, we have no hope of
handling an arbitrary number of repeating $(01)$ in a generic manner.

To make progress, we must instead rewrite the tape as
\begin{equation}
    0^\infty\; (111)\; 1 \rhead B 010101\; (01)\; 1\; 0^\infty.
\end{equation}
After 12 steps of simulation, we reach
\begin{equation}
    0^\infty\; (111)\; 1110110 \rhead D (01)\; 1\; 0^\infty.
\end{equation}
Now, being in state $D$ and with a 0 to the left of the tape head, we have
$0 \rhead D 01 \vdash^* 110 \rhead D$, and so we reach
\begin{equation}
    0^\infty\; (111)\; 111011\; (11)\; 0 \rhead D 1\; 0^\infty,
\end{equation}
and after one step, we have
\begin{equation}
    0^\infty\; (111)\; 111011\; (11)\; 00 \rhead D 0^\infty.
\end{equation}
Since this is a special case of \eqref{eqn:initial-symbolic}, we can
conclude that this machine will never halt. \qed
\end{example}

We formalize this way of reasoning as follows:

\begin{definition} 
    A \emph{tape chunk} is either a symbol $s \in \{0, 1\}$,
    or a repeater $(s)$, with $s \in \{0, 1\}^*$, $s \neq \epsilon$.
    The set of all tape chunks is denoted $\mathcal C$.
\end{definition}

\begin{definition}
    A \emph{symbolic configuration} consists of a directed tape head, and
    possibly empty sequences of tape chunks on either side, of the form
\begin{align}
    0^\infty\; l_1\; \cdots\; l_m &\lhead q r_1\; \cdots\; r_n\; 0^\infty
\shortintertext{or}
    0^\infty\; l_1\; \cdots\; l_m &\rhead q r_1\; \cdots\; r_n\; 0^\infty
\end{align}
where $l_k, r_k \in \mathcal C$.
\end{definition}

\begin{definition}
    A configuration $c$ \emph{matches} a symbolic configuration $F$,
    written $c \in L(F)$, if one can choose a number of repeatitions for each
    repeater in $F$ and obtain $c$.
    \todo{Playing fast and loose here but I haven't thought of any better wording}
\end{definition}

\begin{definition}[Step relation for symbolic configurations]
    We write $F \vdash F'$ if for all $c \in L(F)$ there is a $c' \in L(F')$
    such that $c \vdash c'$. Likewise for $\vdash^*$ and $\vdash^s$.
    \todo{Do we want to use $\vDash$ for this?}
\end{definition}

\begin{definition}[Shift rules]
    If $a \rhead q b\; \vdash^* b'\; a \rhead q$, we say that we have a
    \emph{right-facing shift rule} with tail $a \in \{0, 1\}^*$
    taking $b \in \{0, 1\}^k$ to $b' \in \{0, 1\}^k$.
    Likewise, if $b \lhead q a \,\vdash^*\; \lhead q a\; b'$, we say that we have
    a \emph{left-facing shift rule}.
\end{definition}

\todo{The below is not yet coherent, but merely snippets of what might
be useful, created through endless rewriting}
\begin{remark} The symbolic tapes we're considering can also be understood as
defining a regular language of tapes matching a particular step of the cycle
being simulated. In this interpretation, we are executing the Turing machine
on all elements of the language at the same time.
\end{remark}

We will be considering a tape language of the form
\begin{equation}
    \{
    0^\infty\; \lhead q w_0\; (r_1)^{k_1}\; w_1\; (r_2)^{k_2}\; w_2\: \cdots\: (r_n)^{k_n}\; w_n
    \mid k_1, k_2, \ldots, k_n \in \N
    \}.
\end{equation}
for a fixed list of \emph{walls} $w_0,\ \ldots,\ w_n \in \{0, 1\}^*$,
a list of \emph{repeaters} $r_1,\ \ldots,\ r_n$, and a fixed state $q$.
